\documentclass{article}

% Language setting
% Replace `english' with e.g. `spanish' to change the document language
\usepackage[english]{babel}

% Set page size and margins
% Replace `letterpaper' with `a4paper' for UK/EU standard size
\usepackage[letterpaper,top=2cm,bottom=2cm,left=3cm,right=3cm,marginparwidth=1.75cm]{geometry}

% Useful packages
\usepackage{amsmath}
\usepackage{float}
\usepackage{graphicx}
\usepackage[colorlinks=true, allcolors=blue]{hyperref}

\title{Računalniška grafika zapiski}
\author{Tim Hajdinjak}

\begin{document}
\maketitle

\section{Matematične osnove}

Osnovni pojmi, brez katerih žal ne gre

\begin{itemize}
\item stolpčna matrika: $\begin{bmatrix} 2,9 \\ -4,6 \\ 0 \end{bmatrix}$
\item vrstična matrika:  $\begin{bmatrix} 12,5 & -9,32 & 0 \end{bmatrix}$
\item transponiranje: pomeni zamenjavo osi dveh matrik: $\begin{bmatrix} 1,3 & -4,1 & 0,0 \end{bmatrix}^T = \begin{bmatrix} 1,3 \\ -4,1 \\ 0,0 \end{bmatrix}$
\item enakost matrik: dve matriki sta si enaki, če imata enako število elementov po obeh oseh, in so vsi vsebovani elementi na enakih mestih
\item vektor: predstavljen kot matrika, pomeni premik iz točke v točko in vektorji nimajo lokacije
\item seštevanje matrik: $a + b = c \iff c_i = a_i + b_i$, intuitivno: 
\begin{figure}[H]
\centering
\includegraphics[width=25mm]{src/sestevanje_vektorjev.png}
\caption{Geometrijsko sestevanje vektorjev. (iz repa vektorja 1 na glavo vektorja 2)}
\end{figure}
\item enota za seštevanje: $a + 0 = 0 + a = a \iff \begin{bmatrix} 3 & -1 \end{bmatrix} + \begin{bmatrix} 0 & 0 \end{bmatrix} = \begin{bmatrix} 3 & -1 \end{bmatrix}$
\item odštevanje matrik: $a - b = c \iff c_i = a_i - b_i$
\item množenje s skalarjem: $\alpha * a = b \iff b_i = \alpha * a_i$
\item inverz za seštevanje: $a - a = a + -a = a + (-1)a=0 \iff \begin{bmatrix} 2 & 5 \end{bmatrix} - \begin{bmatrix} 2 & 5 \end{bmatrix} = \begin{bmatrix} 2 & 5 \end{bmatrix} + \begin{bmatrix} -2 & -5 \end{bmatrix}=\begin{bmatrix} 0 & 0 \end{bmatrix}$
\item NORMA oz. dolžina vektorja: $h = \begin{bmatrix} x \\ y \end{bmatrix} \Longrightarrow \lVert h \rVert = \sqrt{x^2 + y^2}, oz. \lVert a \rVert = \sqrt{\sum_{i=1}^n a_i^2}$ (L2 norm oz. evklidska razdalja). Splošna norma: $\lVert a \rVert_p = \sqrt[p]{\sum_{i=1}^n |a_i|^p}$. Torej, če na izpitu reče druga splošna norma, namesto p pišeš 2. Neskončna norma $\to$ dolžina se bliža maksimalni vrednosti vektorja. 
\item enotski vektor, pomeni vektor dolžine 1, $\lVert e \rVert = 1$
\item normalizacija: $v = \begin{bmatrix} v_x \\ v_y \\ v_z \end{bmatrix} \Longrightarrow \hat{v} = v / \lVert v \rVert = \begin{bmatrix} v_x/\lVert v \rVert \\ v_y/\lVert v \rVert \\ v_z/\lVert v \rVert \end{bmatrix}, \hat{v} = $ enotskost vektorja 
\item skalarni produkt: $u = \begin{bmatrix} u_0 \\ u_1 \\ u_2 \end{bmatrix}, v = \begin{bmatrix} v_0 \\ v_1 \\ v_2 \end{bmatrix} \Longrightarrow u * v = u_0*v_0 + u_1*v_1 + u_2*v_2$. Pravimo mu tudi "detektor pravokotnosti", saj:
\begin{enumerate}
    \item $u*v = 0 \to$ vektorja pravokotna
    \item $u*v < 0 \to$ iztegnjen kot
    \item $u*v > 0 \to$ ostri kot
    \item $\hat{a} * \hat{b} \to$ vrednosti med $[-1, 1],-1 = 180^\circ, 0 = 90^\circ, 1 = 0^\circ$
\begin{figure}[H]
\centering
\includegraphics[width=100mm]{src/ortogonalnost.png}
\caption{2 vektorja sta ortogonalna, če je skalarni produkt enak 0 (oz. sta si pravokotna)}
\end{figure}
\begin{figure}[H]
\centering
\includegraphics[width=100mm]{src/skalarni_produkt_enotskih_vektorjev.png}
\caption{Skalarni produkt enotskih vektorjev}
\end{figure}
\end{enumerate}
Še par pravil oz. posebnosti pri skalarnih produktih:
\begin{itemize}
    \item $u * v = \lVert u \rVert * \lVert v \rVert * \cos{\alpha}$
    \item $v * v = \lVert v  \rVert^2$, norma
    \item $u * 0 = 0 * u = 0$, skalarni produkt z vektorjem 0
    \item $0 * 0 = 0$, skalarni produkt vektorja 0
    \item $u * v = v * u$, komutativnost
    \item $u * (v+w) = u*v + u*w$, distributivnost za seštevanje
    \item $(\alpha u)*v = u * (\alpha v) = \alpha * (u * v)$, homogenost za množenje s skalarjem
    \item $u \perp v \iff u * v = 0$, skalarni produkt ortogonalnih (pravokotnih) vektorjev
    \item asociativnost: nedefinirana operacija
\end{itemize}
\item linearna neodvisnot, projekcija vektorja na vektor: $kv = \lVert w \rVert (w_u * v_u)*v_u$
\begin{figure}[H]
\centering
\includegraphics[width=100mm]{src/projekcija_vektorja_na_vektor.png}
\caption{Projekcija vektorja na vektor}
\end{figure}
Postopek pri linearne neodvisnost:
\begin{enumerate}
    \item Izračunaj dolžine vektorjev: $\lVert w \rVert = w * w, \lVert v \rVert = v * v$ 
    \item Izračunaj enotske vektorje: $w_u = w/ \lVert w \rVert, v_u = v/ \lVert v \rVert$
    \item Izračunaj kosinus kota med vektorji: $\cos{\alpha} = w_u * v_u$
    \item Združi v projekcijo: $kv = \lVert w \rVert (w_u * v_u)*v_u$
    \item Izračunaj ortogonalni vektor: $u = w - kv$
\end{enumerate}
\item Vektorski produkt: $u = \begin{bmatrix} u_x \\ u_y \\ u_z \end{bmatrix}, v = \begin{bmatrix} v_x \\ v_y \\ v_z \end{bmatrix}, u \times v = \begin{bmatrix} u_yv_z - u_zv_y \\ u_zv_x - u_xv_z \\ u_xv_y - u_yv_x \end{bmatrix}$
\begin{figure}[H]
\centering
\includegraphics[width=100mm]{src/vektorski_produkt.png}
\caption{Vektorski produkt intuitivno}
\end{figure}
\begin{figure}[H]
\centering
\includegraphics[width=30mm]{src/enotski_vektorji.png}
\caption{Enotski vektorji, ki tvorijo prostor $R^3$, imajo normo (dolžino) 1 in so vzajemno pravokotni. Pri tem je $e_x = (1,0,0), e_y = (0,1,0), e_z = (0,0,1)$}
\end{figure}
Zakonitosti pri vektorskem produktu:
\begin{enumerate}
    \item $u \times v = -(v \times u)$, antikomutativnost
    \item $u \times (v + w) = u \times v + u \times w$, distributivnost za seštevanje
    \item $(\alpha u) \times v = u \times (\alpha v) = \alpha(u \times v)$, homogenost za množenje s skalarjem
    \item Asociativnost ne obstaja:  $u \times (v \times w) \not = (u \times v) \times w$
    \item $u \parallel v \iff u \times v = 0$, vektorski produkt kolinearnih vektorjev
    \item $u \times 0 = 0 \times v = 0$, vektorski produkt vektorja 0
    \item $0 \times 0 = 0$, vektorski produkt z vektorjem 0
    \item $e_x \times e_y = e_z, e_y \times e_z = e_x, e_z \times e_x = e_y$, vektorski produkt koordinatnih osi, zanimivost: vidimo lahko desno pravilo
\end{enumerate}
\item Splošna matrika, notacija: $A_{m \times n} = \begin{bmatrix}
    a_{11} & a_{12} & \cdots & a_{1n} \\
    a_{21} & a_{22} & \cdots & a_{2n} \\
    \vdots & \vdots & \ddots & \vdots \\
    a_{m1} & a_{m2} & \cdots & a_{mn}
\end{bmatrix}$
\item Seštevanje matrik: $A + B = C \iff c_{ij} = a_{ij} + b_{ij}$, primer: $\begin{bmatrix} 2 & 0 \\ -1 & 2 \\ 3 & 5 \end{bmatrix} + \begin{bmatrix} 
 1 & 3 \\ -1 & 2 \\2 & -1 \end{bmatrix} = \begin{bmatrix} 3 & 3 \\ -2 & 4 \\ 5 & 4 \end{bmatrix}$
 Zakonitosti pri seštevanju splošnih matrik:
 \begin{enumerate}
     \item Možno le, če sta matriki enakih dimenzij! 
     \item $A + B = B + A$, komutativnost
     \item $(A + B) + C = A + (B + C)$, asociativnost
     \item $A + 0 = 0 + A = A$, enota za seštevanje
     \item $A - A = A + (-1)A = 0$, inverz za seštevanje
 \end{enumerate}
\item Množenje matrik s skalarjem: $\alpha A = B \iff b_{ij} = \alpha a_{ij}$, primer: $3 * \begin{bmatrix} 2 & 0 \\ -1 & 2 \\ 3 & 5 \end{bmatrix} = \begin{bmatrix} 6 & 0 \\ -3 & 6 \\ 9 & 15 \end{bmatrix}$
Zakonitosti pri množenju matrik s skalarjem:
\begin{enumerate}
    \item $\alpha (A + B) = \alpha A + \alpha B$, distributivnost seštevanja matrik
    \item $(\alpha + \beta)A = \alpha A + \beta A$, distributivnost seštevanja skalarjev
    \item $(\alpha \beta)A = \alpha (\beta A)$, asociativnost
    \item $(-1)A = -A$, množenje s skalarjem -1
\end{enumerate}
\item Množenje matrik: $A_{n \times m}B_{m \times p} = C_{n \times p} \iff c_{ij} = \sum_{k=1}^m a_{ik}b_{kj}$
\begin{figure}[H]
\centering
\includegraphics[width=40mm]{src/mnozenje_matrik.png}
\caption{Miselni vzorec za množenje splošnih matrik}
\end{figure}
Zakonitosti pri množenju matrik:
\begin{enumerate}
    \item $AB \not = BA$, komutativnost ne velja
    \item $(AB)C = A(BC)$, asociativnost
    \item $A(B + C) = AB + AC$, distributivnost za seštevanje
    \item $(A + B)C = AC + BC$, distributivnost za seštevanje
    \item $(\alpha A)B = A(\alpha B) = \alpha (AB)$, homogenost za množenje s skalarjem
    \item $0A = 0$, množenje s skalarjem 0
    \item $A0 = 0A = 0$, množenje z matriko 0
\end{enumerate}
\item Enota za množenje oz. identiteta: $I_n = \begin{bmatrix} 
1 & 0 & 0 & \cdots & 0 \\ 0 & 1 & 0 & \cdots & 0 \\ 0 & 0 & 1 & \cdots & 0 \\ \vdots & \vdots & \vdots & \ddots & \vdots \\ 0 & 0 & 0 & \cdots & 1 \end{bmatrix}_{n \times n}$, 1 po diagonali
\begin{enumerate}
    \item $AB = BA = I \iff B = A^{-1}$
    \item $(ABC)^{-1} = C^{-1}B^{-1}A^{-1}$, inverz za množenje
\end{enumerate}
\item Transponiranje: $A^T = B \iff b_{ij} = a_{ji}$
Lastnosti transponiranja:
\begin{enumerate} 
    \item $(A^T)^T = A$
    \item $(\alpha A)^T = \alpha A^T$
    \item $(A + B)^T = A^T + B^T$
    \item $(ABC)^T = C^TB^TA^T$
    \item $(A^{-1})^T = (A^T)^{-1}$
\end{enumerate}
\end{itemize}

\section{Transformacije in homogene koordinate}

\begin{itemize}
    \item Teselacija je matematični koncept, ki se nanaša na prekrivanje ravnine z enakimi ali različnimi geometrijskimi oblikami brez prekrivanja. Te oblike so lahko trikotniki, kvadrati, šestkotniki,..., ki se ponavljajo na urejen način, da zapolnijo celotno površino. Več oblik dodajamo, bolj natančen bo naš objekt. 
    \item Mozaičenje: razbitje površine na manjše koščke.
    \item LOD (Level of Detail), tehnika, kjer se uporabljajo različne različice 3D modela z različnimi stopnjami podrobnosti glede na razdaljo modela od kamere. Torej, ko je objekt blizu kamere, se uporablja višji nivo (več teselacije), ko pa je daleč, pa nižji (manj teselacije). Namen LOD je izboljšati učinkovitost izrisa ter optimizacijo delovanja sistema. 
    \item Ko pa aplikacija oz. program določi natančnost objektov, jih game engine uredi tako, da zmanjša število podatkov (manj teselacije).
    \item Vedno delamo z ogljišči, vse ostalo je potem posledica.
    \item Linearna transformacija: $p' = f(p)$, točko premaknemo drugam
    \item Razteg/skaliranje: enakomeren oz. neenakomeren
    \begin{figure}[H]
    \centering
    \includegraphics[width=100mm]{src/razteg_skaliranje.png}
    \caption{Primer enakomernega ter neenakomernega skaliranja}
    \end{figure}
    \item Striženje: spreminjamo eno izmed osi  na podlagi vrednosti druge osi. Če želimo striženje za kot $\phi$, uporabimo kotangens kota.
    \begin{figure}[H]
    \centering
    \includegraphics[width=40mm]{src/strizenje.png}
    \caption{Primer striženja}
    \end{figure} 
    \item Zrcaljenje: zrcaljenje čez koordinatne osi. Če zrcalimo dvakrat, dobimo isto. Če želimo zrcalit čez os $x = y$, uporabimo: $p_x' = p_y$ in $p_y' = p_x$
    \begin{figure}[H]
    \centering
    \includegraphics[width=75mm]{src/zrcaljenje.png}
    \caption{Primer zrcaljenja preko Y osi}
    \end{figure} 
    \item Vrtenje: vedno vrtimo v nasprotno smer urinega kazalca okrog izhodišča. 
    \begin{figure}[H]
    \centering
    \includegraphics[width=75mm]{src/vrtenje.png}
    \caption{Primer vrtenja okrog izhodišča}
    \end{figure}     
    Izpeljava:
    \begin{enumerate}
        \item Zapis s polarnimi koordinatami: \\
        $p_x = \lVert p \rVert \cos{\phi}$ \\
        $p_y = \lVert p \rVert \sin{\phi}$
        \item Vrtenje v polarnih koordinatah: \\
        $p_x' = \lVert p \rVert \cos{(\phi + \theta)}$ \\
        $p_y' = \lVert p \rVert \sin{(\phi + \theta)}$
        \item Adicijski izreki kotnih funkcij: \\
        $p_x' = \lVert p \rVert \cos{\phi} \cos{\theta} - \lVert p \rVert \sin{\phi} \sin{\theta}$ \\
        $p_y' = \lVert p \rVert \cos{\phi} \sin{\theta} + \lVert p \rVert \sin{\phi} \cos{\theta}$
        \item Pretvorba v kartezične koordinate: \\
        $p_x' = p_x \cos{\theta} - p_y \sin{\theta}$ \\
        $p_y' = p_x \sin{\theta} + p_y \cos{\theta}$
    \end{enumerate}
    \item Linearne transformacije, zapisane kot matrike:
    \begin{itemize}
        \item Nova točka = matrika $*$ točka
        \item Matrika 2x2 za 2D
        \item Matrika 3x3 za 3D
    \end{itemize}
    \begin{enumerate}
        \item $p' = Mp$, nova točka = matrika $*$ točka
    \end{enumerate}
\end{itemize}


\end{document}